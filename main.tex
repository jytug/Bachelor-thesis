\documentclass[licencjacka]{pracamgr}
\usepackage[MeX]{polski}
%\usepackage[T1]{fontenc}
\usepackage[utf8]{inputenc}
\usepackage{amssymb}
\usepackage{amsmath}
\usepackage{amsthm}

\theoremstyle{definition}
\newtheorem{definition}{Definicja}[section]

\theoremstyle{definition}
\newtheorem{remark}{Uwaga}[section]

\theoremstyle{definition}
\newtheorem{proposition}{Stwierdzenie}[section]

\theoremstyle{definition}
\newtheorem{example}{Przykład}[section]

\theoremstyle{definition}
\newtheorem{corollary}{Wniosek}[section]

\theoremstyle{plain}
\newtheorem{lemma}{Lemat}[section]

\theoremstyle{plain}
\newtheorem{theorem}{Twierdzenie}[section]

\author{Filip Binkiewicz}

\nralbumu{332069}

\title{Własność A dla kompleksów kostkowych $\text{CAT(0)}$}

\tytulang{Property A for $\text{CAT(0)}$ cube complexes}

\opiekun{prof. dr hab. Sławomira Nowaka\\
			Instytut Matematyki\\
		}

\kierunek{Matematyka}

\date{Czerwiec 2015}

\dziedzina{
11.0 Matematyka, Informatyka:\\
11.1 Matematyka\\
}

\klasyfikacja{14 Algebraic Geometry\\
	}

\keywords{
	Kompleks kostkowy $ \text{CAT(0)} $, własność A
}

\newtheorem{defi}{Definicja}[section]

\def\quotient#1#2{%
    \raise1ex\hbox{$#1$}\Big/\lower1ex\hbox{$#2$}%
}

\begin{document}
\maketitle

%%streszczenie - strona początkowa

\begin{abstract}
%	{
	Praca ta skupia się na dowodzie, iż kompleksy kostkowe $\text{CAT(0)} $
	mają własność A.
%	}
\end{abstract}

\tableofcontents

\chapter*{Motywacja}
\addcontentsline{toc}{chapter}{Motywacja}

Motywacja bpeaasdgdagafg

\chapter{Wprowadzenie}

Pierwszy rozdział tej pracy poświęcę przypomnieniu podstawowych definicji, 
twierdzeń i przykładów dotyczących jej tematu. Aby zachować ciągłość pracy, 
postaram się uniknąć przytaczania rozległych dowodów. Dla zainteresowanych 
w odpowiednich miejscach znajdą się odsyłacze do literatury.

\section{Własność A}

Własność A jest pewnym przeniesieniem pojęcia średniowalności na przestrzenie metryczna. 
Przed właściwym wprowadzeniem tego pojęcia przypomne kilka podstawowych 
definicji dotyczących geometrii zgrubnej.

Przez $ X,Y $ będziemy oznaczać przestrzenie metryczne, $ d $ będzie oznaczać metrykę 
pochodzącą z przestrzeni, z której pochodzą jej argumenty. Jeśli będzie to konieczne, 
przez $ d_X, d_Y $ będziemy dla ścisłości oznaczać metryki pochodzące odpowiednio z $ X $ 
i $ Y $.

\begin{definition}
	Powiemy, że funkcja $ \varphi: X \rightarrow Y $ jest \textbf{zgrubna}, jeśli 
	spełnia następujące dwa warunki:
	\begin{itemize}
	\item (\textit{Bornologiczność}) Dla każdego $ R > 0 $ istnieje $ S > 0 $ 
	takie, że 
	$$ d(x_1,x_2) < R \Rightarrow d \left( \varphi(x_1), \varphi(x_2)\right) < S $$
	\item (\textit{Właściwość}) Dla każdego $ S > 0 $ istnieje $ R > 0 $ takie, że 
	$$ d\left(\varphi(x_1), \varphi(x_2)\right) < S \Rightarrow d(x_1,x_2) < R $$
	\end{itemize}
\end{definition}

\begin{example}
	Zanurzenie $ \mathbb{Z} \hookrightarrow \mathbb{R} $ jest zgrubne. Każde 
	przekształcenie liniowe $ \mathbb{Z} \rightarrow \mathbb{Z}, ~ n \rightarrow
	an + b$ jest zgrubne. Przekształcenie $ \mathbb{Z} \ni n \rightarrow n^2 \in 
	\mathbb{Z} $ nie jest zgrubne, bo nie jest bornologiczne ($ d(n, n+1) = 1 $, a 
	$ d(n^2, n^2 + 2n + 1) = |2n+1| $ jest dowolnie duże).
\end{example}

Powiemy, że dwa przekształcenia $ f_1, f_2 : X \rightarrow Y $ są blisko, 
jeśli istnieje $ C > 0 $ takie, że $$ d(f_1(x), f_2(x)) < C  ~ \text{dla każdego } x \in X$$ 
Zbiór $ A \subset X $ jest $ r $-gęsty, jeśli dla każdego $ x \in X $ istnieje 
element $ a \in A $ taki, że $ d(x,a) < r $. Zbiór $ A $ jest zgrubnie gęsty, jeśli 
jest $ r $-gęsty dla pewnego $ r > 0 $.

\begin{definition}
	Powiemy, że przestrzenie $ X,Y $ są \textbf{zgrubnie równoważne}, jeśli 
	istnieją przekształcenia zgrubne $ \varphi : X \rightarrow Y, 
	~ \psi : Y \rightarrow X  $ takie, że 
	$ \varphi \circ \psi $ jest blisko $ \text{id}_Y $, zaś $ \psi \circ \varphi $ jest blisko 
	$ \text{id}_X $. Przestrzenie $ X,Y $ są zgrubnie równoważne wtedy i tylko wtedy, gdy 
	istnieje $ \varphi : X \rightarrow Y $ takie, że $ \varphi(X) \subset Y $ jest 
	podzbiorem zgrubnie gęstym.
\end{definition}

\begin{remark}
	Każda przestrzeń metryczna $ X $ zawiera dyskretny podzbiór zgrubnie gęsty.
\end{remark}
\begin{proof}
	Ustalmy $ \varepsilon > 0 $. Niech $ \mathcal{D} = \{D \subset X : \forall_{x_1, x_2 \in D, ~
	x_1 \neq x_2} ~ d(x_1, x_2) > \varepsilon \} $. Rodzina $ \mathcal{D} $ jest niepusta 
	oraz każdy łańcuch jest ograniczony z góry przez swoją sumę. Wobec lematu Kuratowskiego-
	Zorna istnieje więc maksymalny element $D_0 \in \mathcal{D}$. Jest on $ \varepsilon $-
	gęstym podzbiorem $ X $. Istotnie, załóżmy przeciwnie - istnieje $ x \in X $ taki, że 
	$ d(x, D_0) > \varepsilon $. Wtedy zbiór $ D_0 \cup \{x\} $ należy do rodziny 
	$ \mathcal{D} $ i zawiera w sobie $ D_0 $, co przeczy maksymalności $ D_0 $
\end{proof}

\begin{definition}
	Przestrzeń dyskretna $ X $ ma \textbf{własność A}, jeśli dla każdego $ R>0 $ oraz 
	$ \varepsilon > 0 $ istnieje rodzina niepustych, skończonych zbiorów $ A_x \subset 
	X \times \mathbb{N} $ indeksowana $ x \in X $ oraz stała $ S > 0 $ taka, że 
	spełnione są następujące warunki:
	\begin{enumerate}
		\item Dla każdych dwóch $ x,x' \in X $ zachodzi
		$$ d(x,x') < R \Rightarrow \frac{\# (A_x \Delta A_{x'})}{\# A_x} < \varepsilon$$
		\item Dla dowolnego elementu $(x', n) \in A_x$ zachodzi $$ d(x,x') \leq S $$
	\end{enumerate}
	Dowolna przestrzeń metryczna ma własność A, jeśli zawiera zgrubnie gęsty podzbiór 
	o tej własności.
\end{definition}
Symbol $ \# $ oznacza liczbę elementów zbioru, zaś symbol $ \Delta $ - operację 
różnicy symetrycznej (a więc $A \Delta B = (A \setminus B) \cup (B \setminus A)$)


\begin{remark}
	Własność A jest niezmiennikiem zgrubnej równoważności przestrzeni dyskretnych. 
	Dokładniej, jeśli przestrzenie dyskretne $ X,Y $ są zgrubnie równoważne, to 
	$ X $ ma własność A wtedy i tylko wtedy, gdy $ Y $ ma własność A.
\end{remark}

Uwaga ta jest bezpośrednią konsekwencją poniższego lematu:

\begin{lemma}
	Jeśli $ \varphi: X \rightarrow Y $ jest przekształceniem zgrubnym przestrzeni dyskretnych 
	oraz $ Y $ ma własność A, to $ X $ ma własność A.
\end{lemma}
\begin{proof}
	Łatwo sprawdzić, że istnieje funkcja $\psi: Y \rightarrow X $ taka, że
	$$ d(y, \varphi(\psi(y)))  \leq d(y,  \varphi(X)) \text{ dla każdego } y \in Y$$
	W tym celu wystarczy dla każdego $ y \in Y $ wybrać $ x \in X $ taki, że 
	$ \varphi(x) $ jest odpowiednio blisko $ y $ i ustalić $ x = \psi(y) $.

	Ustalmy teraz $ R > 0,~  \varepsilon > 0 $. Przekształcenie $\varphi$ jest zgrubne, zatem 
	istnieje $ R_0 $ takie, że $$ d(x,x') < R \Rightarrow d(\varphi(x), \varphi(x')) < R_0 $$ 
	Dla stałych $ R_0, \varepsilon $ istnieje rodzina 
	$ \{B_y \subset Y \times \mathbb{N}\}_{y \in Y} $ 
	indeksowana $ y \in Y $ oraz stała $S'$ spełniającaewarunki definicji własności $ A $.
 	 Zdefiniujmy 
	teraz 
	$$ X \times \mathbb{N} \supset A_x = \{(x', n): ~ n \leq \# \{ (y,m) \in B_{\varphi(x)} : ~
	\psi(y) = x' \}\} $$

	Sprawdzimy, że rodzina ta spełnia warunki definicji 1.3.3. Jeśli 
	$ d(x,x') < R $, to $ d(\varphi(x), \varphi(x')) < R' $, a więc 
	$$ \frac{\# (A_x \Delta A_{x'})}{\# A_x} \leq \frac{\# B_{\varphi(x)} \Delta 
	B_{\varphi(x')}}{\# B_{\varphi(x)}} < \varepsilon $$

	Załóżmy wreszcie, że $ (x', n) \in A_x $. Wówczas istnieje para $ (y,m) \in 
	B_{\varphi(x)} $ taka, że $ \psi(y) = x' $. Wówczas $d(y,\varphi(x)) \leq S'$ oraz 
	$$ d(\varphi(x), \varphi(x')) \leq d(\varphi(x),y) + d(y,\varphi(x')) = d(\varphi(x),y) + 
	d(y, \varphi(\psi(y))) \leq 2S' + 1 $$

	Korzystając znów ze zgrubności $\varphi$, możemy znaleźć stałą $ S $ taką, aby 
	$$ d(\varphi(x), \varphi(x')) < 2S' +1 \Rightarrow d(x,x') < S $$
	Otrzymujemy więc, że $ d(x,x') < S $, co kończy dowód.
\end{proof}

W dalszych rozważaniach będziemy korzystać z następującej charakteryzacji własności A:

\begin{proposition}
	Dyskretna przestrzeń metryczna $ X $ ma własność A wtedy i tylko wtedy, gdy 
	istnieje ciąg rodzin funkcji o skończonym nośniku $ f_{n,x} : X \rightarrow \mathbb{N}
	\cup \{0\} $, indeksowany $ x \in X $, oraz ciąg $ S_n \in \mathbb{R}_{+} $ taki, że
	\begin{enumerate}
		\item Dla każdego $ n $ oraz $ x $ nośnikiem $ f_{n,x} $ jest $ B(S_n, x) $.
		\item Dla każdego $ R > 0 $ ciąg $$ \frac{\| f_{n,x} - f_{n,x'} \|}{\| f_{n,x} \|} $$
		zbiega jednostajnie do zera na zbiorze $ \{ (x,x'): ~ d(x,x') \leq R\} $ przy 
		$ n \rightarrow \infty $. Norma $ \| \cdot \| $ oznacza normę $ \ell_1 $ na 
		przestrzeni funkcji na o skończonym nośniku określonych na $ X $.
	\end{enumerate}
\end{proposition}
\begin{proof}
	Powyższe warunki są równoważne z następującym: dla każdego $ R>0, ~ \varepsilon > 0 $ 
	istnieje rodzina funkcji o skończonym nośniku $ f_x : X \rightarrow \mathbb{N} \cup 
	\{0\}$, indeksowana $ x \in X $, oraz $ S > 0 $ takie, że $ \text{supp}{f_x} = B(S,x)  $ 
	oraz $$ d(x,x') \leq R \Rightarrow \frac{\| f_x - f_{x'} \|}{\|f_x \|} < \varepsilon $$

	Konieczność tego warunku wynika stąd, że przekształcenie $ f_x(y) = \#\left( A_x \cap \left( 
	\{y\} \times \mathbb{N} \right) \right) $ spełnia powyższe warunki, zaś dostateczność - 
	stąd, iż $ A_x = \{(y,n) \in X \times \mathbb{N} : ~ 1 \leq n \leq f_x(y) \} $ spełnia 
	warunki definicji 1.3.3.
\end{proof}



\section{Przestrzenie $\text{CAT(0)}$}
Niech $ (X, d) $ będzie przestrzenią metryczną. Odcinkiem geodezyjnym nazywamy 
przekształcenie izometryczne $ \mathbb{R} \supset I \xrightarrow{\rho} X $, gdzie 
$ I =[a,b]$ jest odcinkiem. Przestrzeń $ X $ nazwiemy (jednoznacznie) geodezyjną, 
jeśli każde dwa punkty można połączyć (jednoznacznie wyznaczonym) odcinkiem 
geodezyjnym.

\begin{example}
	Każda przestrzeń euklidesowa $ \mathbb{R}^n $ jest jednoznacznie geodezyjna, 
	jak również każdy jej wypukły podzbiór. Sfera $ S^2 $ jest geodezyjna, ale 
	nie jednoznacznie - dwa bieguny można połączyć ścieżką geodezyjną na 
	nieskończenie wiele sposobów. Każdy spójny graf metryczny jest przestrzenią 
	geodezyjną.
\end{example}

Dalej będziemy rozważać przestrzenie geodezyjne. Dla wygody przez $ [x,y] $ będziemy 
oznaczać (dowolny) odcinek geodezyjny łączący $ x \in X$ z $ y \in X $ (a dokładniej 
obraz tego odcinka).

Zwróćmy uwagę, że jeśli $ X $ jest przestrzenią geodezyjną, to dla każdej trójki 
$ (x,y,z) \in X^3 $ istnieje trójka $ (\overline{x},\overline{y}, \overline{z}) 
\in \left(\mathbb{R}^2\right)^3$ taka, że $ d(x,y) = d_{\mathbb{R}^2} (\overline{x}, 
\overline{y}), ~ d(x,z) = d_{\mathbb{R}^2} (\overline{x}, \overline{z}), 
~ d(y,z) = d_{\mathbb{R}^2} (\overline{y}, \overline{z})  $. Innymi słowy, każdemu 
trójkątowi z $ X $ można przypisać trójkąt z przestrzeni euklidesowej $ \mathbb{R}^2 $ o 
bokach takiej samej długości. Taki trójkąt jest wyznaczony jednoznacznie z dokładnością 
do izometrii przestrzeni $ \mathbb{R}^2 $ i nazwiemy go trójkątem porównania $ (x,y,z) $.

\begin{definition}
	Powiemy, że przestrzeń geodezyjna $ X $ jest $ \text{\textbf{CAT(0)}} $, jeśli 
	dla każdej trójki $ (x,y,z) \in X^3 $ oraz punktu $ p \in [y,z] $ oraz odpowiadającym 
	im trójkątowi porównania $ (\overline{x}, \overline{y}, \overline{z}) \in 
	\left(\mathbb{R}^2\right)^3 $ i punktowi $ \overline{p} \in [\overline{y}, 
	\overline{z}] $ zachodzi nierówność:
	$$ d(x,p) \leq d_{\mathbb{R}^2}(\overline{x}, \overline{p}) $$
\end{definition} 

Innymi słowy, w przestrzeniach $ \text{CAT(0)} $ trójkąty są ,,szczuplejsze'' niż w 
przestrzeni euklidesowej. O takich przestrzeniach powiemy, że mają niedodatnią 
krzywiznę.

%%miejsce na rysuneczek chudego trójkąta

\begin{example} Nietrudno jest o kilka przykładów takich przestrzeni:
	\begin{itemize}
	\item Każda przestrzeń euklidesowa $ \mathbb{R}^n $ jest $ \text{CAT(0)} $. Wówczas 
	wymieniona nierówność jest po prostu równością.
	\item Graf metryczny jest przestrzenią $ \text{CAT(0)} $ wtedy i tylko wtedy, gdy 
	jest drzewem.
	\end{itemize}
\end{example}

\begin{remark}
	Każda przestrzeń $ \text{CAT(0)} $ jest jednoznacznie geodezyjna.
\end{remark}
\begin{proof}
	Przypuśćmy przeciwnie i niech $ x,y \in X $ łączą dwa różne odcinki geodezyjne, 
	powiedzmy $ [x,y], \overline{[x,y]} $.	Wówczas istnieją $ [x,y] \ni p \neq \overline p 
	\in	\overline{[x,y]}  $ takie, że $ d(x,p) = d(x,\overline{p}) $ oraz $ d(y,p) = 
	d(y, \overline{p}) $. Wówczas trójkątowi $ (x,y,\overline{p}) $ w $ \mathbb{R}^2$ 
	odpowiada trójkąt zdegenerowany, zaś $ d(p,\overline{p}) > 0 $, co przeczy nierówności 
	$ \text{CAT(0)} $
\end{proof}

\begin{corollary}
	Sfera $ S^2 $ nie jest przestrzenią $ \text{CAT(0)} $. Płaszczyzna $ \mathbb{R}^2 $ 
	wyposażona w metrykę pochodzącą od normy $ \ell_1 $ nie jest przestrzenią $ \text{CAT(0)} $
	%miejsce na rysuneczek płaszczyzny z l_1
\end{corollary}

\section{Kompleksy kostkowe $ \text{CAT(0)} $}
Niech $ K = [0,1]^n $ będzie $ n $-wymiarową kostką. Będzie to podstawowy ,,budulec'' 
interesujących nas przestrzeni. Przez ścianę o kowymiarze równym 1 będziemy rozumieć zbiór 
$$ F_{i,\varepsilon}  = \{x \in K : ~ x_i = \varepsilon\}, ~ \text{ dla } i = 1 
\dots n \text{ oraz }\varepsilon \in \{0,1\}$$

Wszystkie ściany o niższym kowymiarze (o wyższym wymiarze) można otrzymać jako 
przecięcie ścian o wyższym kowymiarze.

\begin{definition}
	Niech $ K,K' $ będą dwiema kostkami oraz $ F \subset K, ~ F' \subset K' $ będą 
	ich ścianami. \textbf{Sklejeniem} (lub \textbf{przyłączeniem}) $ K $ z $ K' $ nazwiemy 
	izometrię $ \varphi: F \rightarrow F' $.
\end{definition}

\begin{definition}
	Przypuśćmy, że $ \mathcal{K} $ jest zbiorem kostek (dla każdego $ K \in 
	\mathcal{K} $ istnieje $ n(K) \in \mathbb{N} $ takie, że $ K \simeq [0,1]^{n(K)} $), 
	zaś $ \mathcal{S} $ - zbiorem sklejeń elementów $ \mathcal{K} $ (każdemu 
	$ \varphi \in \mathcal{S} $ odpowiadają kostki $ K = K(\varphi), K' = 
	K'(\varphi) \in \mathcal{K} $ oraz ściany $ F \subset K, F' \subset K' $. Załóżmy 
	wreszcie, że taka para 	$ (\mathcal{K}, \mathcal{S})  $ spełnia następujące warunki:

	\begin{enumerate}
		\item Żadna kostka nie jest sklejona sama ze sobą.
		\item Dla każdych dwóch kostek $ K \neq K' $ istnieje co najwyżej jedno
		sklejenie $ K $ z $ K'$.
	\end{enumerate}

	Wówczas w następujący sposób można zdefiniować \textbf{kompleks kostkowy}:
	$$ X =  \quotient{\left( \bigsqcup\limits_{K \in \mathcal{K}} C \right)}{\sim} $$

	gdzie $ \sim $ dla każdego $ \varphi \in \mathcal{S} $ utożsamia dziedzinę $ \varphi $ 
	z jego obrazem, to znaczy: $$ \{ x \sim \varphi(x) ~ | ~ \varphi \in \mathcal{S}, ~ 
	x \in \text{dom}(\varphi) \} $$

	Jeśli istnieje stała $ M > 0 $ taka, że dla każdego $  K  \simeq [0,1]^{n(K)}\in 
	\mathcal{K} $ zachodzi $ n(K) < M $, to kompleks kostkowy $ X $ jest \textbf{skończenie 
	wymiarowy}. Wtedy \textbf{wymiarem} tego kompleksu nazwiemy liczbę 
	$$ \dim X = \max\limits_{K \in \mathcal{K}} n(K)$$
\end{definition}

\begin{remark}
	W ten sposób zdefiniowany kompleks kostkowy jest przestrzenią metryczną, przy czym 
	metryka długości\footnote{length metric} indukowana jest z metryki euklidesowej 
	na $ [0,1]^n \subset \mathbb{R}^n$. Odległość punktów $ x,y $ mierzona w metryce 
	długości jest to infinum długości krzywych $ \gamma : [a,b] \rightarrow X $ 
	łączących $ x$ z $ y $. Długość krzywej definiujemy następująco: 
	$$ l(\gamma) = \sup\limits_{a = t_0 \leq \dots \leq t_n = b} \sum\limits_{i=0}^{n-1}
	d(\gamma(t_i), \gamma(t_{i+1})) $$
\end{remark}
\begin{proposition}
	Z powyższej definicji łatwo wynikają następujące fakty:
	\begin{itemize}
	\item Obcięcie rzutowania $ p : \bigsqcup_{K \in \mathcal{K}} \rightarrow X$ do jednej 
	kostki $ K \in \mathcal{K}$ jest iniekcją.
	\item Niepuste przecięcie dwóch kostek jest ścianą obydwu.
	\end{itemize}	
\end{proposition}

\begin{example}
	Łatwo o kilka prostych przykładów kompleksów kostkowych:
	\begin{itemize}
	\item Rozważmy graf metryczny bez wierzchołków izolowanych, w którym każda krawędź ma 
	długość 1. Każda krawędź jest izometryczna z $ [0,1] $, zaś sklejenia to
	po prostu izometrie punktów. Jest to prosty przykład kompleksu kostkowego.
	\item Torus można interpretować jako kompleks kostkowy. Rozważmy zbiór 
	$ [0,3]\times[0,3] \subset \mathbb{R}^2 $, w którym można wprowadzić podział 
	na dziewięć części izometrycznych z $ [0,1]^2 $. Wtedy odpowiednie izometrie 
	prowadzą do konstrukcji torusa (rysunek).
	%miejsce na rysuneczek torusa
	\end{itemize}
\end{example}


\begin{remark}
	Na zbiorze wierzchołków można wprowadzić metrykę długości krawędziowej, gdzie 
	odległość dwóch wierzchołków to minimum długości łączących ich ścieżek złożonych 
	z krawędzi kompleksu (przez krawędź rozumiemy ścianę o wymiarze 1).
	Z naszego punktu widzenia możemy utożsamić te metryki, z uwagi na następujący fakt:
\end{remark}

\begin{proposition}
	Niech $ X $ będzie kompleksem kostkowym $ \text{CAT(0)}$. Metryka długości na 
	zbiorze wierzchołków $ X $ jest zgrubnie równoważna z metryką długości krawędziowej.
	Jeśli $ X $ jest skończenie wymiarowy, to zbiór wierzchołków z pierwszą bądź drugą z 
	tych metryk jest sobie zgrubnie równoważny.
\end{proposition}
\begin{proof}
%WTF?
%	Na potrzeby dowodu przez $ d_1 $ oznaczymy metrykę długości, zaś przez $ d_2 $ - 
%	metrykę długości krawędziowej. Niech 
%	$$ x = x_0, x_1, x_2 \dots x_n = y$$
%	będzie ciągiem wierzchołków na najkrótszej ścieżce krawędziowej łączącej $ x $ oraz $ y $.
%	Wówczas
%	$$ d_1(x,y) \leq \sum\limits_{i = 0}^{n-1} d_1(x_{i}, x_{i+1}) = n = d_2(x,y) $$
%
%	Odwrotnie, mając dwa wierzchołki $ x,y $ takie, że $ d_2(x,y) = k $, to odcinek 
%	łączący $ x $ oraz $ y $ jest kompleksem kostkowym $ \text{CAT(0)} $, który 
%	zanurza się jako podkompleks $ [0,1]^k $. Zanurzenie to jest izometrią ze względu na 
%	$ d_2 $ kontrakcją ze względu na $ d_1 $. Obraz punktu $ x $ przy tym zanurzeniu 
%	oznaczmy przez $ \tilde x $ i, nadużywając nieco notacji, przez $ d_1, d_2 $ oznaczmy 
%	również odpowiednio metrykę długości oraz długości krawędziowej również w kostce 
%	$ [0,1]^k$. Wtedy
%	$$ d_2(x,y) = d_2(\tilde x, \tilde y) \sqrt{d_1(\tilde x, \tilde y)} \leq 
%	\sqrt{d_1(x,y)} $$
\end{proof}

Kompleksy kostkowe $ \text{CAT(0)} $ posiadają strukturę kombinatoryczną. Zbiór 
wierzchołków drogowo spójnego kompleksu kostkowego $ \text{CAT(0)} $ można podzielić na dwa 
drogowo spójne podzbiory zbioru wierzchołków. Taki podział nazwiemy 
\textbf{hiperpłaszczyzną}, zaś oba 
spójne podzbiory nazwiemy \textbf{półprzestrzenią}. 

%CO ZA DEBIL TO PISAŁ!!!!
%
%Trzeba wpisać dobrą definicję hiperpłaszczyzny
%

Dwie hiperpłaszczyzny tworzą podział kompleksu na cztery przecięcia półpłaszczyzn. Jeśli 
wszystkie są niepuste, to hiperpłaszczyzny \textbf{przecinają się}. Dwa wierzchołki 
$ x,y $ są \textbf{oddzielone} przez hiperpłaszczyznę $ H $, jeśli należą do różnych 
wyznaczonych przez nią półprzestrzeni. 

Zbiór hiperpłaszczyzn oddzielających $ x $ od $ y $ będziemy oznaczać przez 
$\mathfrak{H}(x,y)$. \textbf{Odcinkiem} łączącym $ x $ oraz $ y $ nazwiemy przecięcie 
wszystkich półprzestrzeni zawierających obydwa te punkty i oznaczymy $ [x,y]$. Zbiór 
wierzchołków $ V $ nazwiemy \textbf{wypukłym}, jeśli dla każdych $ x,y \in V $ również 
$ [x,y] \subset V $.

Dla trzech wierzchołków $ w,x,y $ możemy wyróżnić ich \textbf{medianę}, zdefiniowaną 
jako jedyny wierzchołek należący do $ [w,x] \cap [x,y] \cap [w,y] $

Dla kompleksu kostkowego $ \text{CAT(0)}~ X$ możemy wprowadzić brzeg kombinatoryczny. 
Niech funkcja $ \sigma $ przypisuje hiperpłaszczyźnie $ X $ jedną z wyznaczonych przez nią 
półprzestrzeni, przy czym dla każdych dwóch hiperpłaszczyzn $ H_1, H_2 $ zachodzi 
$ \sigma(H_1) \cap \sigma(H_2) \neq \emptyset $. Taką funkcję nazwiemy \textbf{ultrafiltrem}.

Wierzchołek $ x $ definiuje takie przekształcenie: dla hiperpłaszczyzny $ H $ wyznacza 
półprzestrzeń $ H_x $ zawierającą $ x $ (rzeczywiście, dla każdych dwóch hiperpłaszczyzn 
$ H,K $ mamy $ x \in H \cap K $). Jeśli więc oznaczymy przez $\mathfrak{U}$ zbiór wszystkich 
ultrafiltrów na $ X $, zaś przez $ V (X) $ - zbiór wierzchołków $ X $, to wskazaliśmy iniekcję 
$$ \iota : V(X) \rightarrow \mathfrak{U} $$

Wówczas elementy zbioru $$ \partial X = \mathfrak{U} \setminus \iota(V(X)) $$
nazwiemy \textbf{krawędziami w nieskończoności}. Utożsamiając z wierzchołkiem $ x $ 
ultrafiltr $ \iota(x) $, możemy więc zdefiniować
$$ \overline{X} = X \cup \partial X $$
Powyższy zbiór nazwiemy \textbf{dopełnieniem w nieskończoności} kompleksu $ X $.

Możemy przenieść podstawowe kombinatoryczne własności 
kompleksu kostkowego $ \text{CAT(0)}$ 
na jego dopełnienie w nieskończoności. Jeśli $ z,w \in \overline{X} $, to dla 
hiperpłaszczyzny $ H $ przez $ H_z, H_w $ będziemy oznaczać obraz $ z,w $ 
(jako ultrafiltrów) na 
$ H $, a więc odpowiednią półprzestrzeń (wtedy powiemy, że $ H_z $ zawiera $ z $. 
Hiperpłaszczyzna $ H $ \textbf{oddziela} $ x $ od $ w $, 
jeśli $ H_z \neq H_w $. Można więc na $ \overline{X} $ uogólnić definicję zbioru 
$ \mathfrak{H}(x,w)$. Podobnie możemy zdefiniować odcinek $ [x,w] $ jako 
$$ [x,w] = \bigcap H_{x,w}  ~~ x,w \in H_{x,w}, ~ H_{x,w} - \text{ półprzestrzeń}$$

Zwróćmy uwagę, że każdy odcinek $ [x,w] $ jest wypukły. Wynika to stąd, że przecięcie 
zbiorów wypukłych takie jest. Oczywiste jest również następujące stwierdzenie: 

\begin{proposition}
	Niech $ x,y,w \in X $ oraz $ x \in \overline{X} $. Jeśli $ w \in [x,z] $ oraz 
	$ y \in [y,w] $, to $ \mathfrak{H}(y,w) \subset \mathfrak{H}(y,z)$
\end{proposition}

\begin{remark}
	Na zbiorze $ \overline{X} $ trudno wprowadzić metrykę, można natomiast w naturalny 
	sposób zrobić z niego przestrzeń topologiczną. Powiemy, że ciąg wierzchołków 
	$ \{x_j\}_{j = 1}^{\infty} \subset X $ zbiega do wierzchołka $ x \in \overline{X} $, jeśli 
	dla każdej hiperpłaszczyzny $ H $ zachodzi $ H \in \mathfrak{H}(x_j, x) $ jedynie 
	dla skończenie wielu $ j $. Piszemy wówczas, że $$ x_j 
	\xrightarrow{j \rightarrow \infty} x $$
\end{remark}

Sekcję tę zakończymy serią lematów i twierdzeniem łączącym kompleks 
kostkowy z przestrzenią euklidesową.

\begin{lemma}
	Niech $ \{x_j\}_{j = 1}^{\infty} \subset X, x \in \overline{x} $ oraz niech 
	$ x_j \rightarrow x$ przy $ j \rightarrow \infty $. Hiperpłaszczyzna $ H $ 
	oddziela $ y $ od $ x $ wtedy i tylko wtedy, gdy oddziela $ y $ od prawie 
	wszystkich\footnote{wszystkich, oprócz skończenie wielu} $ x_j $. Inaczej:
	$$ \mathfrak{H}(y,x) = \bigcup\limits_{k = 1}^{\infty} \bigcap\limits_{j = k}^{\infty}
	\mathfrak{H}(y,x_j) $$
\end{lemma}
\begin{proof}
	Wystarczy wspomnieć definicję: $ H $ oddziela $ y $ od $ x $ wtedy i tylko wtedy, gdy 
	$ H_y \neq H_z $, a więc wtedy i tylko wtedy, gdy $H_y \neq H_{x_j} $ dla prawie 
	wszystkich $ j \in \mathbb{N} $. 
\end{proof}
\begin{lemma}
	Niech $ \{x_j\}_{j = 1}^{\infty} \subset X, x \in \overline{X} $ oraz niech 
	$ x_j \rightarrow x$ przy $ j \rightarrow \infty $. Ponadto niech $ y,z \in X $. 
	Wówczas jeden \underline{i tylko jeden} z poniższych warunków jest prawdziwy.
	\begin{itemize}
		\item $ y \in [z,x_j] $ dla prawie wszystkich $ j \in \mathbb{N} $
		(wtedy $ y \in [z,x]$).
		\item $ y \notin [z,x_j] $ dla prawie wszystkich $ j \in \mathbb{N} $
		(wtedy $ y \notin [z,x] $).
	\end{itemize}
\end{lemma}
\begin{proof}
	Negacją warunku pierwszego jest warunek: $ y \notin [z,x_j] $ dla nieskończenie 
	wielu $ j \in \mathbb{N} $. Wynika on łatwo z drugiego warunku. Pokażemy, że warunek 
	drugi jest mu równoważny.
	
	Jeśl $ y \notin [z,x_j]  $, to istnieje $ H \in \mathfrak{H}(y,z) $ taka, że 
	$ H_z = H_{x_j} $. %%whyyyyyyy.
	Jeśli jest tak dla nieskończenie wielu $ j $, to ze skończoności zbioru 
	$ \mathfrak{H}(y,z) $ wynika, że istnieje hiperpłaszczyzna 
	$ H \in \mathfrak{H}(y,z)$ taka, że $ H_{z} = H_{x_j} $ dla nieskończenie wielu $ j \in 
	\mathbb{N} $. Zatem $ H_z = H_x = H_{x_j} $ dla prawie wszystkich $ j \in \mathbb{N} $. 
	W szczególności $ y \notin [z,x_j] $ dla niemal wszystkich $ j $, a więc 
	$ y \notin [z,x] $.
	
	Pozostaje wykazać, że pierwszy warunek pociąga za sobą, że $ y \in [z,x] $.
	Załóżmy że $ y \notin [z,x] $. Istnieje więc hiperpłaszczyzna  
	$ H \in \mathfrak{H}(y,z) $ taka, że $ H_x = H_z $, a więc $ H_{x_j} = H_{z} $ dla 
	niemal wszystkich $ j \in \mathbb{N} $. A więc $ y \notin [z,x_j] $ dla 
	prawie wszystkich $ j \in \mathbb{N} $ i otrzymujemy sprzeczność.
\end{proof}

\begin{lemma}
	Niech $ x,y \in X $ oraz $ z \in \overline{X} $. Wówczas przecięcie odcinków 
	$ [x,y], [x,z], [y,z] $ składa się z pojedynczego wierzchołka z $ X $.
\end{lemma}
\begin{proof}
	Najpierw wykażemy, że przecięcie to jest niepuste, następnie - że ma tylko jeden element.

	Niech $ \{z_j\}_{j = 1}^{\infty} \subset X $ będzie ciągiem wierzchołków zbieżnym 
	do $ z $. Odcinek $[x,y] $ jest skończony i zawiera mediany $ m_j = m(x,y,z_j) $. 
	Istnieje więc $ m \in [x,y] $ taki, że $ m = m_j \in [x,z_j] $ dla nieskończenie 
	wielu $ j \in \mathbb{N} $. Z poprzedniego lematu wynika więc, że $ m \in [x,z_j] $ dla 
	niemal wszystkich $ j \in \mathbb{N} $ oraz $ m \in [x,z] $. Podobnie $ m \in [y,z] $.

	Załóżmy teraz, że $ m \neq m' $ należą do $ [x,y] \cap [x,z] \cap [y,z] $ i niech 
	$ H \in \mathfrak{H}(m,m') $. Któreś dwie z półprzestrzenie $ H_x, H_y, H_z $ są 
	sobie równe; dla ustalenia uwagi niech $ H_x = H_y $. Skoro $ H_m \neq H_{m'} $, 
	tylko jedno z nich może być równe $ H_x $, zatem znowu dla ustalenia uwagi niech 
	$ H_m \neq H_x $. Wtedy $ m \notin [x,z] $, co daje sprzeczność.
\end{proof}

\begin{remark}
	W powyższym dowodzie skorzystaliśmy z faktu, że mamy ciąg $X \ni z_j \rightarrow z \in 
	\overline{X}$. 
	Istnienie takiego ciągu nie jest zupełnie oczywiste - aby je uzasadnić, należy 
	rozważyć zbiór wszystkich hiperpłaszczyzn $ H_1, H_2, \dots $ (jest on przeliczalny) 
	oraz dla każdego $ j \in \mathbb{N}$ wybrać $ z_j $ należące do zbioru
	$$ \bigcap\limits_{i=1}^j (H_i)_z $$
	
	Aby uzasadnić, że powyższy zbiór jest niepusty, wystarczy skorzystać z twierdzenia 
	Helly'ego mówiącego o przecięciach zbiorów wypukłych. %rol98, Helly's theorem
\end{remark}
Dla $ x \in X, z \in \overline{X} $ przez $ \mathfrak{R}_z(x) $ oznaczymy 
podzbiór $ \mathfrak{H}(x,z) $ złożony z tych hiperpłaszczyzn, które oddzielają 
$x $ od $ z $ oraz pewnego sąsiada $ x $. 
\begin{lemma}
	Niech $ X $ będzie kompleksem kostkowym $ \text{CAT(0)} $ oraz $ \dim{X} < \infty $. 
	Ponadto niech $ x \in X, z \in \overline{X} $.
	Wówczas $ \# \mathfrak{R}_z(x) \leq \dim{X}$
\end{lemma}
\begin{proof}
	Dowód można znaleźć w (BCGNW)
	 lemat 1.13
\end{proof}

W następnym twierdzeniu uzasadnimy, że odcinki łączące wierzchołki (być może 
w nieskończoności) zanurzają się w odpowiednio dużą przestrzeń euklidesową. W oczywisty 
sposób $ \mathbb{R}^d $ możemy postrzegać jako kompleks kostkowy (patrz przykład 1.3.1). 
Zbiorem wierzchołków jest krata $ \mathbb{Z}^d $. Odcinkami są prostopadłościany - 
dokładniej, jeśli 
$ \overline{x} = (x_1, \dots, x_d), \overline{y} = (y_1, \dots, y_d)  \in\mathbb{Z}^d $, 
to odcinkiem $ [\overline{x}, \overline{y}] $ jest powłoka wypukła podzbioru $ \mathbb{Z}^d$ 
złożonego z tych liczb, których $ i $-ta współrzędna jest z przedziału $ [x_i, y_i] $ lub
$ [y_i, x_i] $.
Żeby włączyć do naszych rozważań wierzchołki w nieskończoności, dopuszczamy możliwość, że 
$ x_i  $ lub $ y_i = \pm \infty $ dla pewnego $ i = 1 \dots d$.

\begin{theorem} \label{thm:zanurzanie}
	Niech $ X $ będzie skończenie wymiarowym kompleksem kostkowym $ \text{CAT(0)} $, 
	$ \dim{X} = d $ oraz niech $ x,y \in \overline{X} $. Wówczas odcinek $ [x,y] $ zanurza 
	się izometrycznie w kompleksie kostkowym $ \mathbb{R}^d $.
\end{theorem}

Wprowadźmy na zbiorze $ \mathfrak{H}(x,y) $ częściowy porządek w następujący sposób:
$$ H \preceq K \iff H_x \subset K_x $$

\begin{lemma}
	Dwie płaszczyzny $H,K \in \mathfrak{H}(x,y)$ nie są porównywalne w tym 
	porządku wtedy i tylko wtedy, gdy się przecinają.
\end{lemma}
\begin{proof}
	Oczywiste jest, że $ H_x \cap K_x \neq \emptyset \neq H_y \cap K_y $. Dalej, 
	$ H_x \cap K_y = \emptyset \iff H_x \subset K_x  \wedge H_y \cap K_x = \emptyset \iff
	K_x \subset H_x$. $ H $ oraz $ K $ są więc nieporównywalne przez $ \preceq $ wtedy 
	i tylko wtedy, gdy $ K_x \not\subset H_x $ oraz $ H_x \not\subset K_x $, a więc 
	wtedy, gdy wszystkie cztery przecięcia są niepuste.
\end{proof}

\begin{lemma}[Dilworth]
	Niech $ (S, \preceq) $ będzie zbiorem częściowo uporządkowanym. Łańcuchem nazwiemy 
	podzbiór $S $, 
	którego elementy są parami porównywalne, antyłańcuchem - podzbiór $ S $ nieposiadający 
	dwóch różnych elementów porównywalnych. Jeśli zbiór $ S $ nie zawiera antyłańcucha 
	o mocy $ m + 1 $, to $ S $ jest sumą rozłączną $ m $ łańcuchów.
\end{lemma}
\begin{proof}
	Można znaleźć w (Dilworth, a decomposition theorem for partially ordered sets).
\end{proof}
\begin{corollary}
	Zbiór częściowo uporządkowany $ \left(\mathfrak{H}(x,y), \preceq \right) $ 
	jest sumą rozłączną $ d $ 
	łańcuchów.	
\end{corollary}
\begin{proof}
	Wystarczy skorzystać z twierdzenia Helly'ego, lematu Dilwortha oraz 
	lematu 1.3.5 
\end{proof}
\begin{proof}[Dowód twierdzenia 1.3.1]
	Dowód tego twierdzenia przeprowadzimy tylko dla przypadku, gdy $ x $ jest 
	wierzchołkiem $ X $. Weźmy rozkład zbioru $ \mathfrak{H}(x,y)$ na łańcuchy, którego 
	istnienia dostarcza poprzedni lemat
	 $$ \mathfrak{H}(x,y) = \bigsqcup\limits_{i=1}^{d} \mathfrak{B}_i  $$
	Niech teraz 
	$$ X \supset [x,y] \ni z \rightarrow \overline{z} = (\overline{z_1}, \dots, 
	\overline{z_d}) \in \mathbb{Z}^d, ~ \overline{z_i} = \# \{H \in \mathfrak{B}_i: ~ 
	z \in H_y\} $$

	Wówczas $ \overline{x} = 0$, zaś $ \overline{y} = (\overline{y_1}, \dots, 
	\overline{y_d}) $, gdzie $ \overline{y_i} = \# \mathfrak{B}_i, i = 1 \dots d$. 
	Dla każdego $ z \in [x,y] $ współrzędne $ \overline{z} $ są skończone oraz 
	$ \overline{z} \in [\overline{x}, \overline{y}] $.

	Funkcja $ z \rightarrow \overline{z} $ zanurzeniem izometrycznym. Żeby to sprawdzić, 
	wystarczy obliczyć:
	$$ d(\overline{v}, \overline{w}) = \sum\limits_{i=1}^d \# \{ 
	H \in \mathfrak{B}_i: ~ H \in \mathfrak{H}(v,w) \} = \# \mathfrak{H}(v,w) = 
	d(v,w)$$
\end{proof}

\section{Kombinacje}
Funkcje, spełniające warunki Stwierdzenia 1.1.1. będziemy konstruować przy 
użyciu dwumianu Newtona, a więc funkcji $ {n \choose k} $. Kombinatorycznie funkcja ta 
oznacza \textit{liczbę $ k $-podzbiorów $ n $-zbioru}, w szczególności jej definicja jest 
poprawna dla całkowitych $ n \geq k \geq 0 $. Przy użyciu łatwej interpretacji 
kombinatorycznej można udowodnić relację rekurencyjną:

\begin{itemize}
	\item $ {n \choose 0} = {n \choose n} = 1 $ dla $ n \geq 0 $.
	\item $ {n \choose k} = {n-1 \choose k-1} + {n-1 \choose k} $
\end{itemize}

Można łatwo uogólnić tę funkcję dla wszystkich $ n,k \in \mathbb{Z} $ poprzez relację:

\begin{itemize}
	\item $ {n \choose 0} = 1  $ dla $ n \geq 0$ oraz $ {n \choose n} = 1  $ dla 
	$ n \in \mathbb{Z}$
	\item $ {n \choose k} = {n -1 \choose k-1} + {n-1 \choose k} $ dla wszystkich $ n,k \in
	\mathbb{Z} $
\end{itemize}

Z powyższej definicji łatwo udowodnić kilka własności:

\begin{itemize}
	\item ${n \choose k} = 0$ dla $ k < 0 \leq n $
	\item $ {n \choose k} = (-1)^{n+k}{-1-r \choose -1-n} $
\end{itemize}

Szczególnie przydatna okaże się druga własność, dla $ k = -1 $. Przyjmuje wtedy ona formę 
$$ {n \choose -1} = (-1)^{n-1} {0 \choose -1-n} $$

\chapter{Kompleksy kostkowe $ \text{CAT(0)} $ a własność A}

\end{document}
