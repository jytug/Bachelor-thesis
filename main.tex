\documentclass[licencjacka]{pracamgr}
\usepackage[MeX]{polski}
%\usepackage[T1]{fontenc}
\usepackage[utf8]{inputenc}
\usepackage{amssymb}
\usepackage{amsmath}
\usepackage{amsthm}

\theoremstyle{definition}
\newtheorem{definition}{Definicja}[section]

\theoremstyle{definition}
\newtheorem{remark}{Uwaga}[section]

\theoremstyle{definition}
\newtheorem{example}{Przykład}[section]

\theoremstyle{definition}
\newtheorem{corollary}{Wniosek}[section]

\theoremstyle{plain}
\newtheorem{lemma}{Lemat}[section]

\theoremstyle{plain}
\newtheorem{theorem}{Twierdzenie}[section]

\author{Filip Binkiewicz}

\nralbumu{332069}

\title{Własność A dla kompleksów kostkowych $\text{CAT(0)}$}

\tytulang{Property A for $\text{CAT(0)}$ cube complexes}

\opiekun{prof. dr hab. Sławomira Nowaka\\
			Instytut Matematyki\\
		}

\kierunek{Matematyka}

\date{Czerwiec 2015}

\dziedzina{
11.0 Matematyka, Informatyka:\\
11.1 Matematyka\\
}

\klasyfikacja{14 Algebraic Geometry\\
	}

\keywords{
	Kompleks kostkowy $ \text{CAT(0)} $, własność A
}

\newtheorem{defi}{Definicja}[section]

\def\quotient#1#2{%
    \raise1ex\hbox{$#1$}\Big/\lower1ex\hbox{$#2$}%
}

\begin{document}
\maketitle

%%streszczenie - strona początkowa

\begin{abstract}
%	{
	Praca ta skupia się na dowodzie, iż kompleksy kostkowe $\text{CAT(0)} $
	mają własność A.
%	}
\end{abstract}

\tableofcontents

\chapter*{Motywacja}
\addcontentsline{toc}{chapter}{Motywacja}

Motywacja bpeaasdgdagafg

\chapter{Wprowadzenie}

Pierwszy rozdział tej pracy poświęcę przypomnieniu podstawowych definicji, 
twierdzeń i przykładów dotyczących jej tematu. Aby zachować ciągłość pracy, 
postaram się uniknąć przytaczania rozległych dowodów. Dla zainteresowanych 
w odpowiednich miejscach znajdą się odsyłacze do literatury.

\section{Przestrzenie $\text{CAT(0)}$}
Niech $ (X, d) $ będzie przestrzenią metryczną. Odcinkiem geodezyjnym nazywamy 
przekształcenie izometryczne $ \mathbb{R} \supset I \xrightarrow{\rho} X $, gdzie 
$ I =[a,b]$ jest odcinkiem. Przestrzeń $ X $ nazwiemy (jednoznacznie) geodezyjną, 
jeśli każde dwa punkty można połączyć (jednoznacznie wyznaczonym) odcinkiem 
geodezyjnym.

\begin{example}
	Każda przestrzeń euklidesowa $ \mathbb{R}^n $ jest jednoznacznie geodezyjna, 
	jak również każdy jej wypukły podzbiór. Sfera $ S^2 $ jest geodezyjna, ale 
	nie jednoznacznie - dwa bieguny można połączyć ścieżką geodezyjną na 
	nieskończenie wiele sposobów. Każdy spójny graf metryczny jest przestrzenią 
	geodezyjną.
\end{example}

Dalej będziemy rozważać przestrzenie geodezyjne. Dla wygody przez $ [x,y] $ będziemy 
oznaczać (dowolny) odcinek geodezyjny łączący $ x \in X$ z $ y \in X $ (a dokładniej 
obraz tego odcinka).

Zwróćmy uwagę, że jeśli $ X $ jest przestrzenią geodezyjną, to dla każdej trójki 
$ (x,y,z) \in X^3 $ istnieje trójka $ (\overline{x},\overline{y}, \overline{z}) 
\in \left(\mathbb{R}^2\right)^3$ taka, że $ d(x,y) = d_{\mathbb{R}^2} (\overline{x}, 
\overline{y}), ~ d(x,z) = d_{\mathbb{R}^2} (\overline{x}, \overline{z}), 
~ d(y,z) = d_{\mathbb{R}^2} (\overline{y}, \overline{z})  $. Innymi słowy, każdemu 
trójkątowi z $ X $ można przypisać trójkąt z przestrzeni euklidesowej $ \mathbb{R}^2 $ o 
bokach takiej samej długości. Taki trójkąt jest wyznaczony jednoznacznie z dokładnością 
do izometrii przestrzeni $ \mathbb{R}^2 $ i nazwiemy go trójkątem porównania $ (x,y,z) $.

\begin{definition}
	Powiemy, że przestrzeń geodezyjna $ X $ jest $ \text{\textbf{CAT(0)}} $, jeśli 
	dla każdej trójki $ (x,y,z) \in X^3 $ oraz punktu $ p \in [y,z] $ oraz odpowiadającym 
	im trójkątowi porównania $ (\overline{x}, \overline{y}, \overline{z}) \in 
	\left(\mathbb{R}^2\right)^3 $ i punktowi $ \overline{p} \in [\overline{y}, 
	\overline{z}] $ zachodzi nierówność:
	$$ d(x,p) \leq d_{\mathbb{R}^2}(\overline{x}, \overline{p}) $$
\end{definition} 

Innymi słowy, w przestrzeniach $ \text{CAT(0)} $ trójkąty są ,,szczuplejsze'' niż w 
przestrzeni euklidesowej. O takich przestrzeniach powiemy, że mają niedodatnią 
krzywiznę.

%%miejsce na rysuneczek chudego trójkąta

\begin{example} Nietrudno jest o kilka przykładów takich przestrzeni:
	\begin{itemize}
	\item Każda przestrzeń euklidesowa $ \mathbb{R}^n $ jest $ \text{CAT(0)} $. Wówczas 
	wymieniona nierówność jest po prostu równością.
	\item Graf metryczny jest przestrzenią $ \text{CAT(0)} $ wtedy i tylko wtedy, gdy 
	jest drzewem.
	\end{itemize}
\end{example}

\begin{remark}
	Każda przestrzeń $ \text{CAT(0)} $ jest jednoznacznie geodezyjna.
\end{remark}
\begin{proof}
	Przypuśćmy przeciwnie i niech $ x,y \in X $ łączą dwa różne odcinki geodezyjne, 
	powiedzmy $ [x,y], \overline{[x,y]} $.	Wówczas istnieją $ [x,y] \ni p \neq \overline p 
	\in	\overline{[x,y]}  $ takie, że $ d(x,p) = d(x,\overline{p}) $ oraz $ d(y,p) = 
	d(y, \overline{p}) $. Wówczas trójkątowi $ (x,y,\overline{p}) $ w $ \mathbb{R}^2$ 
	odpowiada trójkąt zdegenerowany, zaś $ d(p,\overline{p}) > 0 $, co przeczy nierówności 
	$ \text{CAT(0)} $
\end{proof}

\begin{corollary}
	Sfera $ S^2 $ nie jest przestrzenią $ \text{CAT(0)} $. Płaszczyzna $ \mathbb{R}^2 $ 
	wyposażona w metrykę pochodzącą od normy $ \ell_1 $ nie jest przestrzenią $ \text{CAT(0)} $
	%miejsce na rysuneczek płaszczyzny z l_1
\end{corollary}

\section{Kompleksy kostkowe $ \text{CAT(0)} $}
Niech $ K = [0,1]^n $ będzie $ n $-wymiarową kostką. Będzie to podstawowy ,,budulec'' 
interesujących nas przestrzeni. Przez ścianę o kowymiarze równym $ 1 $ będziemy rozumieć 
zbiór $$ F_{i,\varepsilon}  = \{x \in K : ~ x_i = \varepsilon\}, ~ \text{ dla } i = 1 
\dots n \text{ oraz }\varepsilon \in \{0,1\}$$

Wszystkie ściany o niższym kowymiarze (o wyższym wymiarze) można otrzymać jako 
przecięcie ścian o wyższym kowymiarze.

\begin{definition}
	Niech $ K,K' $ będą dwiema kostkami oraz $ F \subset K, ~ F' \subset K' $ będą 
	ich ścianami. \textbf{Sklejeniem} (lub \textbf{przyłączeniem}) $ K $ z $ K' $ nazwiemy 
	izometrię $ \varphi: F \rightarrow F' $.
\end{definition}

\begin{definition}
	Przypuśćmy, że $ \mathcal{K} $ jest zbiorem kostek (dla każdego $ K \in 
	\mathcal{K} $ istnieje $ n(K) \in \mathbb{N} $ takie, że $ K \simeq [0,1]^{n(K)} $), 
	zaś $ \mathcal{S} $ - zbiorem sklejeń elementów $ \mathcal{K} $ (każdemu 
	$ \varphi \in \mathcal{S} $ odpowiadają kostki $ K = K(\varphi), K' = 
	K'(\varphi) \in \mathcal{K} $ oraz ściany $ F \subset K, F' \subset K' $. Załóżmy 
	wreszcie, że taka para 	$ (\mathcal{K}, \mathcal{S})  $ spełnia następujące warunki:

	\begin{enumerate}
		\item Żadna kostka nie jest sklejona sama ze sobą.
		\item Dla każdych dwóch kostek $ K \neq K' $ istnieje co najwyżej jedno
		sklejenie $ K $ z $ K'$.
	\end{enumerate}

	Wówczas w następujący sposób można zdefiniować \textbf{kompleks kostkowy}:
	$$ X =  \quotient{\left( \bigsqcup\limits_{K \in \mathcal{K}} C \right)}{\sim} $$

	gdzie $ \sim $ dla każdego $ \varphi \in \mathcal{S} $ utożsamia dziedzinę $ \varphi $ 
	z jego obrazem, to znaczy: $$ \{ x \sim \varphi(x) ~ | ~ \varphi \in \mathcal{S}, ~ 
	x \in \text{dom}(\varphi) \} $$
\end{definition}

\begin{remark}
	W ten sposób zdefiniowany kompleks kostkowy jest przestrzenią metryczną, przy czym 
	metryka długości\footnote{length metric} indukowana jest z metryki euklidesowej 
	na $ [0,1]^n \subset \mathbb{R}^n$. Odległość punktów $ x,y $ mierzona w metryce 
	długości jest to infinum długości krzywych $ \gamma : [a,b] \rightarrow X $ 
	łączących $ x$ z $ y $. Długość krzywej definiujemy następująco: 
	$$ l(\gamma) = \sup\limits_{a = t_0 \leq \dots \leq t_n = b} \sum\limits_{i=0}^{n-1}
	d(\gamma(t_i), \gamma(t_{i+1})) $$
\end{remark}
\end{document}
