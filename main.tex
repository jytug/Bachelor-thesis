\documentclass[licencjacka]{pracamgr}
\usepackage[MeX]{polski}
%\usepackage[T1]{fontenc}
\usepackage[utf8]{inputenc}
\usepackage{amssymb}
\usepackage{amsmath}
\usepackage{amsthm}

\theoremstyle{definition}
\newtheorem{definition}{Definicja}[section]

\theoremstyle{definition}
\newtheorem{remark}{Uwaga}[section]

\theoremstyle{definition}
\newtheorem{example}{Przykład}[section]

\theoremstyle{plain}
\newtheorem{lemma}{Lemat}[section]

\theoremstyle{plain}
\newtheorem{theorem}{Twierdzenie}[section]

\author{Filip Binkiewicz}

\nralbumu{332069}

\title{Skończony wymiar asymptotyczny kompleksów kostkowych $\text{CAT(0)}$}

\tytulang{Finite asymptotic dimesion for $\text{CAT(0)}$ cube complexes}

\opiekun{prof. dr hab. Sławomira Nowaka\\
			Instytut Matematyki\\
		}

\kierunek{Matematyka}

\date{Czerwiec 2015}

\dziedzina{
11.0 Matematyka, Informatyka:\\
11.1 Matematyka\\
}

\klasyfikacja{14 Algebraic Geometry\\
	 	54F45 Dimension theory\\
	}

\keywords{
	$ \text{CAT(0)} $, wymiar asymptotyczny, kompleks kostkowy
}

\newtheorem{defi}{Definicja}[section]

\begin{document}
\maketitle

%%streszczenie - strona początkowa

\begin{abstract}
%	{
	Celem tej pracy jest udowodnienie, że wymiar asymptotyczny skończenie 
	wymiarowych kompleksów kostkowych $\text{CAT(0)}$ jest ograniczony 
	przez ich wymiar topologiczny.
%	}
\end{abstract}

\tableofcontents

\chapter*{Motywacja}
\addcontentsline{toc}{chapter}{Motywacja}

Motywacja bpeaasdgdagafg

\chapter{Wprowadzenie}

Pierwszy rozdział tej pracy poświęcę przypomnieniu podstawowych definicji, 
twierdzeń i przykładów dotyczących jej tematu. Aby zachować ciągłość pracy, 
postaram się uniknąć przytaczania rozległych dowodów. Dla zainteresowanych 
w odpowiednich miejscach znajdą się odsyłacze do literatury.

\section{Przestrzenie $\text{CAT(0)}$}
Niech $ (X, d) $ będzie przestrzenią metryczną. Odcinkiem geodezyjnym nazywamy 
przekształcenie izometryczne $ \mathbb{R} \supset I \xrightarrow{\rho} X $, gdzie 
$ I =[a,b]$ jest odcinkiem. Przestrzeń $ X $ nazwiemy (jednoznacznie) geodezyjną, 
jeśli każde dwa punkty można połączyć (jednoznacznie wyznaczonym) odcinkiem 
geodezyjnym. 

\begin{example}
	Każda przestrzeń euklidesowa $ \mathbb{R}^n $ jest jednoznacznie geodezyjna, 
	jak również każdy jej wypukły podzbiór. Sfera $ S^2 $ jest geodezyjna, ale 
	nie jednoznacznie - dwa bieguny można połączyć ścieżką geodezyjną na 
	nieskończenie wiele sposobów. Każdy spójny graf metryczny jest przestrzenią 
	geodezyjną.
\end{example}

\begin{definition}
Odcinkiem geodezyjnym
\end{definition} 
\end{document}
