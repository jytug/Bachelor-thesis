\documentclass[licencjacka]{pracamgr}
\usepackage[MeX]{polski}
%\usepackage[T1]{fontenc}
\usepackage[utf8]{inputenc}
\usepackage{amssymb}
\usepackage{amsmath}
\usepackage{amsthm}

\theoremstyle{definition}
\newtheorem{definition}{Definicja}[section]

\theoremstyle{definition}
\newtheorem{remark}{Uwaga}[section]

\theoremstyle{definition}
\newtheorem{example}{Przykład}[section]

\theoremstyle{plain}
\newtheorem{lemma}{Lemat}[section]

\theoremstyle{plain}
\newtheorem{theorem}{Twierdzenie}[section]

\author{Filip Binkiewicz}

\nralbumu{332069}

\title{Skończony wymiar asymptotyczny kompleksów kostkowych $\text{CAT(0)}$}

\tytulang{Finite asymptotic dimesion for $\text{CAT(0)}$ cube complexes}

\opiekun{prof. dr hab. Sławomira Nowaka\\
			Instytut Matematyki\\
		}

\kierunek{Matematyka}

\date{Czerwiec 2015}

\dziedzina{
11.0 Matematyka, Informatyka:\\
11.1 Matematyka\\
}

\klasyfikacja{14 Algebraic Geometry\\
	 	54F45 Dimension theory\\
	}

\keywords{
	$ \text{CAT(0)} $, wymiar asymptotyczny, kompleks kostkowy
}

\newtheorem{defi}{Definicja}[section]

\begin{document}
\maketitle

%%streszczenie - strona początkowa

\begin{abstract}
%	{
	Celem tej pracy jest udowodnienie, że wymiar asymptotyczny skończenie 
	wymiarowych kompleksów kostkowych $\text{CAT(0)}$ jest ograniczony 
	przez ich wymiar topologiczny.
%	}
\end{abstract}

\tableofcontents

\chapter*{Motywacja}
\addcontentsline{toc}{chapter}{Motywacja}

Motywacja bpeaasdgdagafg

\chapter{Wprowadzenie}

Pierwszy rozdział tej pracy poświęcę przypomnieniu podstawowych definicji, 
twierdzeń i przykładów dotyczących jej tematu. Aby zachować ciągłość pracy, 
postaram się uniknąć przytaczania rozległych dowodów. Dla zainteresowanych 
w odpowiednich miejscach znajdą się odsyłacze do literatury.

\section{Przestrzenie $\text{CAT(0)}$}
Niech $ (X, d) $ będzie przestrzenią metryczną. Odcinkiem geodezyjnym nazywamy 
przekształcenie izometryczne $ \mathbb{R} \supset I \xrightarrow{\rho} X $, gdzie 
$ I =[a,b]$ jest odcinkiem. Przestrzeń $ X $ nazwiemy (jednoznacznie) geodezyjną, 
jeśli każde dwa punkty można połączyć (jednoznacznie wyznaczonym) odcinkiem 
geodezyjnym.

\begin{example}
	Każda przestrzeń euklidesowa $ \mathbb{R}^n $ jest jednoznacznie geodezyjna, 
	jak również każdy jej wypukły podzbiór. Sfera $ S^2 $ jest geodezyjna, ale 
	nie jednoznacznie - dwa bieguny można połączyć ścieżką geodezyjną na 
	nieskończenie wiele sposobów. Każdy spójny graf metryczny jest przestrzenią 
	geodezyjną.
\end{example}

Dalej będziemy rozważać przestrzenie geodezyjne. Dla wygody przez $ [x,y] $ będziemy 
oznaczać (dowolny) odcinek geodezyjny łączący $ x \in X$ z $ y \in X $ (a dokładniej 
obraz tego odcinka).

Zwróćmy uwagę, że jeśli $ X $ jest przestrzenią geodezyjną, to dla każdej trójki 
$ (x,y,z) \in X^3 $ istnieje trójka $ (\overline{x},\overline{y}, \overline{z}) 
\in \left(\mathbb{R}^2\right)^3$ taka, że $ d(x,y) = d_{\mathbb{R}^2} (\overline{x}, 
\overline{y}), ~ d(x,z) = d_{\mathbb{R}^2} (\overline{x}, \overline{z}), 
~ d(y,z) = d_{\mathbb{R}^2} (\overline{y}, \overline{z})  $. Innymi słowy, każdemu 
trójkątowi z $ X $ można przypisać trójkąt z przestrzeni euklidesowej $ \mathbb{R}^2 $ o 
bokach takiej samej długości. Taki trójkąt jest wyznaczony jednoznacznie z dokładnością 
do izometrii przestrzeni $ \mathbb{R}^2 $ i nazwiemy go trójkątem porównania $ (x,y,z) $.

\begin{definition}
	Powiemy, że przestrzeń geodezyjna $ X $ jest $ \text{\textbf{CAT(0)}} $, jeśli 
	dla każdej trójki $ (x,y,z) \in X^3 $ oraz punktu $ p \in [y,z] $ oraz odpowiadającym 
	im trójkątowi porównania $ (\overline{x}, \overline{y}, \overline{z}) \in 
	\left(\mathbb{R}^2\right)^3 $ i punktowi $ \overline{p} \in [\overline{y}, 
	\overline{z}] $ zachodzi nierówność:
	$$ d(x,p) \leq d_{\mathbb{R}^2}(\overline{x}, \overline{p}) $$
\end{definition} 

Innymi słowy, w przestrzeniach $ \text{CAT(0)} $ trójkąty są ,,szczuplejsze'' niż w 
przestrzeni euklidesowej. O takich przestrzeniach powiemy, że mają niedodatnią 
krzywiznę.

%%miejsce na rysuneczek chudego trójkąta

\begin{example} Nietrudno jest o kilka przykładów takich przestrzeni:
	\begin{itemize}
	\item Każda przestrzeń euklidesowa $ \mathbb{R}^n $ jest $ \text{CAT(0)} $. Wówczas 
	wymieniona nierówność jest po prostu równością.
	\item Graf metryczny jest przestrzenią $ \text{CAT(0)} $ wtedy i tylko wtedy, gdy 
	jest drzewem.
	\end{itemize}
\end{example}
\end{document}
